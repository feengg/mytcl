\documentclass[11pt,a4paper,twoside,normalheadings,headsepline,DIV13,BCOR11mm,openright]{article}

\usepackage{scrpage2}
\usepackage{a4wide}
\usepackage{url}
\usepackage{graphicx}
\pagestyle{scrheadings}
\clearscrheadfoot
\ihead[]{Bonnmotion}
\chead[]{-\pagemark-}
\ohead[]{Documentation}
\ifoot[]{Institute of Computer Science IV, University of Bonn}
\ofoot[]{[Page \pagemark]}

\usepackage[pdftex,
       bookmarks=true,
       colorlinks=true,
       linkcolor=black,
       citecolor=black,
       urlcolor=black,
       pdftitle={BonnMotion - Manual},
       pdfauthor={Nils Aschenbruck},
       pdfsubject={},
       pdfkeywords={},
       bookmarksopen=false,
       pdfpagemode=UseOutlines
       ]{hyperref}

\begin{document}

\title{BonnMotion -\\a Mobility Scenario Generation and Analysis Tool}
\author{Copyright \copyright  2002-2009 University of Bonn}
\date{\today}

	\thispagestyle{empty}
\begin{titlepage}
	\begin{center}

        \includegraphics[height=30mm]{logo}\\[45mm]
	
	\textbf{\Huge{BonnMotion}}\\[10mm]
        \textbf{\LARGE{a Mobility Scenario Generation and Analysis Tool}}\\[20mm]

	
        \textbf{\Large{Documentation}}\\[7mm]
	\Large{Version: \today}\\[45mm]

	\Large{Copyright \copyright  2002-2009 University of Bonn}\\[22mm]
		  
	\end{center}
\end{titlepage}

\tableofcontents

\clearpage

\section{Legal notice}

This  program is free software; you can redistribute it and/or modify it
under the terms of the GNU General Public License as  published  by  the
Free  Software  Foundation; either version 2 of the License, or (at your
option) any later version.

This program is distributed in the hope that  it  will  be  useful,  but
WITHOUT   ANY   WARRANTY;   without   even   the   implied  warranty  of
MERCHANTABILITY or FITNESS FOR A PARTICULAR PURPOSE. See the GNU General
Public License for more details.

You  should have received a copy of the GNU General Public License along
with this program; if not, write to the Free Software Foundation,  Inc.,
59 Temple Place, Suite 330, Boston, MA  02111-1307  USA

\section{Contact information}

eMail:
\par
\begingroup
\leftskip=0.5cm % Parameter anpassen
\noindent %ab hier der Text, der eingerückt werden soll
Nils Aschenbruck $\langle \textrm{tacnet@cs.uni-bonn.de} \rangle$
\par
\endgroup

\begingroup
\leftskip=0cm % Parameter anpassen
\noindent %ab hier der Text, der eingerückt werden soll
URL:
\par
\endgroup

\begingroup
\leftskip=0.5cm % Parameter anpassen
\noindent %ab hier der Text, der eingerückt werden soll
\url{http://bonnmotion.cs.uni-bonn.de/}
\par
\endgroup

\begingroup
\leftskip=0cm % Parameter anpassen
\noindent %ab hier der Text, der eingerückt werden soll
Postal address:
\par
\endgroup

\begingroup
\leftskip=0.5cm % Parameter anpassen
\noindent %ab hier der Text, der eingerückt werden soll
Nils Aschenbruck\newline
Institute of Computer Science IV, University of Bonn\newline
Roemerstr. 164, 53117 Bonn, Germany\newline
\par
\endgroup


\section{Introduction}


BonnMotion is a Java software which creates and analyses mobility
scenarios. It is developed within the Communication Systems group at
the Institute of Computer Science IV of the University of Bonn,
Germany, where it serves as a tool for the investigation of mobile ad
hoc network characteristics. The scenarios can also be exported for
the network simulators ns-2, GloMoSim/QualNet, COOJA, and MiXiM.


\section{Installation}


To use this software, you need to have a JDK or JRE  installed.  It  has
been compiled and tested with Java v1.5.0\_18.

During  the  installation,  a  few other shell scripts / batch files are
created in the "bin" folder, which you can move and rename as you like:

\begin{itemize}
\item{"bm" is a wrapper that starts the BonnMotion application. Starting it
   without command line parameters prints a detailed help message.}

\item{"compile"  re-compiles the sources. By default, only applies to files
   that were changed after  the  last  compilation.  To  re-compile  all
   sources, use "compile all".}

\item{"makedoc" uses javadoc to create a source code documentation.}
\end{itemize}

This  distribution  includes  the  compiled class files so if you do not
want to modify the code, there is no need for you to run the compile  or
makedoc scripts.


\subsection{Installation on UNIX operating systems}

If  you are using a UNIX platform, simply unpack the archive and run the
"install" script. You will then be asked for the location of  your  Java
binary  path,  i.e.  the  directory  containing the Java interpreter and
possibly the Java compiler and the javadoc utility.

NOTE: We have not yet run our shell  scripts  on  other  platforms  than
Linux  or  Cygwin.  We  would  be happy if you informed us about changes
necessary  to  run  them  under  other  operating  systems  so  we   can
incorporate them into our distribution.


\subsection{Installation on Microsoft Windows operating systems}

NOTE:  The  batch  files  for  MS  Windows  have not yet been thoroughly
tested.

If you are using  Microsoft  Windows,  please  edit  the  two  variables
JAVAPATH and BONNMOTION in the "install.bat" and then execute it.


\section{Running}


All  applications  described  below  are  started  via  the "bm" wrapper
script. The syntax is as follows:
\newline

   bm $\langle \textrm{parameters} \rangle$ $\langle \textrm{application}
   \rangle$ $\langle \textrm{application parameters}\rangle$
\newline \newline
The  application  can  be  a  mobility  model  or  e.g.  the  Statistics
application used to analyse scenario characteristics.

Starting the script without command line parameters prints further help.
Usage examples are given in the sections below.


\section{Scenario generation}

Currently, there are several mobility models available, which are
introduced later on with some comments on implementation details. For
further details on the models themselves, please refer to the
literature. Various synthetic models were proposed during last
decade. There have been several general surveys
\cite{camp_classification_2002,bettstetter_classification_2001,bai_helmy_adhocsensor_ch1,musolesi-mascolo_survey_2008}
as well as some specific ones for vehicular models
\cite{hoogendoorn-survey_vehicular_2001}.

There are two possibilities to feed input parameters into  the  scenario
generation:  The  first  is to enter the parameters on the command line,
and the second is to have a file containing the  parameters.  These  two
methods  can also be combined; in this case, the command line parameters
override those given in the input file.

The scenario generator writes all parameters used to  create  a  certain
scenario  to  a  file.  In  this  way, settings are saved and particular
scenario parameters can be varied without the need to re-enter all other
parameters.

Important  parameters  used  with all models are the following: The node
number is set with -n, the scenario duration (in seconds)  with  -d  and
the -i parameter specifies, how many additional seconds at the beginning
of the scenario should be skipped. With -x and -y, the width and  height
(in metres) of the simulation area are set. With -R, the random seed can
be set manually.

Cutting off the initial phase is an important feature and therefore,  -i
has  a  high  default  value:  It has been observed that with the Random
Waypoint model, nodes have a higher probability of being near the center
of  the  simulation area, while they are initially uniformly distributed
over the simulation area. In our implementation of  the  Manhattan  Grid
model, all nodes start at (0,0) for simplicity.
\newline \newline
Usage examples:
\newline

   bm -f scenario1 RandomWaypoint -n 100 -d 900 -i 3600
\newline

This creates a Random Waypoint scenario with 100 nodes and a duration of
900 seconds. An initial phase of 3600 seconds is cut off.

A scenario is saved in two files: The first, with the suffix  ".params",
contains  the  complete  set  of parameters used for the simulation. The
second, with the suffix ".movements.gz" contains the (gzipped)  movement
data.

Now,  you  can  take  a  look  at the "scenario1.params" file to see all
parameters used for the simulation,  and  if  you  wanted  to  create  a
similar scenario, only with a higher mobility, you can do this with
\newline

   bm -f scenario2 -I scenario1.params RandomWaypoint -h 5.0 
\newline \newline
which  takes the whole parameter set from scenario1.params and overrides
the maximum speed with 5 m/s.


\subsection{The Random Waypoint model ("RandomWaypoint")}


Our  implementation  includes  a  feature  to  restrict   the   mobiles'
movements:  With  "-d 1", nodes move only along the x-axis, with "-d 2",
nodes move either along the x- or along the y-axis (with  a  probability
of  0.5  each)  and  with the default "-d 3", it is the classical Random
Waypoint model as you know it.

Another feature is the "-c" switch that causes positions  to  be  chosen
from  a  circle  that fits into the simulation area rather than from the
simulation area itself.

Instead of choosing new  destinations  uniformly  distributed  from  the
simulation  area,  "attraction  points"  can  be  defined  with the "-a"
parameter, followed by the data characterising  the  attraction  points.
Each attraction point is defined by four floating point numbers, in this
order:  $<$x-coordinate$>$,  $<$y-coordinate$>$,  $<$intensity$>$,   and  
$<$standard deviation$>$.

\begin{itemize}
\item{The coordinates give the attraction point's position.}

\item{The  intensity  levels  weight the attraction points: A point with an
   intensity x times as high as another point's will also attract a node
   with a probability which is x times as high.}

\item{The  last  parameter  is  the  standard  deviation  of  the  Gaussian
   distribution with mean  0  that  is  used  to  determine  the  nodes'
   distances  to  the  attraction  point  on  each of the two dimensions
   (i.e., the distance  to  the  attraction  point  does  not  follow  a
   Gaussian distribution, but a Rayleigh distribution).}
\end{itemize}

The  parameters  for  several attraction points are simply concatenated,
seperated by commas. For example, to place two attraction points on  the
simulation  area,  the  first at (100,100) with intensity 1 and standard
deviation 20, the second at (350,200) with intensity  1.5  and  standard
deviation 31, use "-a 100,100,1,20,350,200,1.5,31".


\subsection{The Manhattan Grid model ("ManhattanGrid")}

The Manhattan Grid model is introduced in \cite{ETSI_Manhattan_Grid_1998}.
In this model, nodes move only on predefined paths. The arguments -u and
-v set the number of blocks between the paths. As an example, "-u  3  -v
2" places the following paths on the simulation area:

\begin{tabular}{ccccccc}
+ & - & + & - & + & - & +\\
$|$ & & $|$ & & $|$ & & $|$\\
+ & - & + & - & + & - & +\\
$|$ & & $|$ & & $|$ & & $|$\\
+ & - & + & - & + & - & +
\end{tabular}
   
Our  implementation  contains  some  (reasonable)  modifications  of the
Manhattan Grid model:

1) An additional parameter we introduce is the minimum speed of a mobile
node.  This  is helpful because the speed of a mobile can be arbitrarily
close to 0 and since the model defines that the speed is to  be  updated
in  \_distance\_  intervals,  there  can be very long periods of very slow
node movement without this parameter.

2) The possibility to have nodes  pause  was  added  with  help  of  two
additional  parameters: The pause probability (if a node does not change
its speed, it will pause with that probability) and  the  maximum  pause
time.

Note  that  it  is especially important to cut off an initial phase with
this model, because in our implementation, all nodes start at  the  same
position (0,0).


\subsection{Gauss-Markov models}
\subsubsection{The original Gauss-Markov model ("OriginalGaussMarkov")}

This implementation of the Gauss-Markov model follows the publication
\cite{Haas_GM_1999}. In this implementation, the mean velocity vector
mu is not specified directly; instead, the norm is specified using
"-a" and a random vector with this norm is assigned to each
station. Of course, a norm of 0 yields only the vector (0,0). The
implementation also allows the user to specify a maximum speed. A
velocity vectors with a larger norm will be multiplied with an
appropriate scalar to reduce the speed to the maximum speed.

The  model  has  been  adapted  to  deal  with  scenario  borders in the
following way: If a station moves onto the border, its  velocity  vector
as well as its expected velocity vector are "mirrored".


\subsubsection{The Gauss-Markov model ("GaussMarkov")}

This is the implementation of the Gauss-Markov model, which rather
follows the description in \cite{camp_classification_2002}, though it
is not the same.  The main commonalites are that for each mobile node,
two seperate values are maintained instead of one speed vector: The
mobile's speed and its direction of movement. Also the default method
of handling mobile nodes that move out of the simulation area is
closely related to \cite{camp_classification_2002}: Nodes may continue
to walk beyond the area boundary, which causes the next movement
vector update not to be based on the prior angle, but on an angle that
brings the nodes back onto the field.  Therefore, the field size is
automatically adapted to the node movements after scenario generation.

The main difference to \cite{camp_classification_2002} is that new
speed and direction of movement are simply chosen from a normal
distribution with a mean of the respective old value (the standard
deviation is specified on the command line using -a and -s). Speed
values are constrained to a certain interval that can be specified on
the command line using -m and -h: If a newly chosen speed value is
outside of this interval, it is changed to the closest value inside of
the interval (which is either the minimum or the maximum value).

The behaviour described above can be modified with several command
line switches: Using -b, the size of the simulation area is fixed and
nodes simply "bounce" at the area boundaries.  Using -u, the speed
values outside of the valid speed interval are adapted in a way that
leads to a uniform distribution of node speeds (instead of peaks
around the interval boundaries).


\subsection{The Reference Point Group Mobility model ("RPGM")}

The implementation of this model \cite{RPGM_99} includes the
possibility to have "dynamic" groups: When a node comes into the area
of another group, it changes to this new group with a probability that
can be set with "-c $<$probability$>$". Deactivate this feature with "-c
0". Note that when this feature is activated, "empty" groups may be
moving along the simulation area and nodes coming into their areas may
change their memberships to these.


\subsection{Static scenarios ("Static")}


By default, nodes in static scenarios are homogeneously distributed over
the  simulation  area.  There  are two possibilities for non-homogeneous
node distributions:

\begin{itemize}
\item Attraction points can be defined with the "-a" parameter; a  detailed
   explanation  of  this  feature  is  given  in  the  "Random Waypoint"
   section.

\item With the "-l" parameter, the simulation  area  can  be  divided  into
   several  areas  with different node densities along its x-axis. Given
   the number n of density leves, each of the n  areas  will  contain  a
   fraction of approximately
      $2 * k / (n * (n + 1)), 1 <= k <= n$,
   of the nodes. (The density decreases from left to right.)
\end{itemize}

   The  following  example shall illustrate this: Distributing 150 nodes
   on a 300m x 300m field with 3 density levels, approx. 75  nodes  will
   lie  within  the rectangle area with the corners (0,0) and (100,300),
   approx. 50 nodes will lie within (100,0) and (200,300) and approx. 25
   nodes will lie within (200,0) and (300,300).

\subsection{Disaster Area model ("DisasterArea")}

Creates scenarios according to \cite{NA_MSWIM2007}.

\begin{itemize}
\item Tactical areas can be defined with the "-b" parameter following a
   list  of  coordinates for the area (seperated by ","). The last three
   values in this list are type, wanted groups and transportgroups.  

\item  Type  can  be  0  for  incident  location, 1 for patients waiting for
   treatment  area,  2  for casualties clearing station, 3 for technical
   operational command and 4 for ambulance parking point.
   
\item  For  every  specified  incident  location  there  must  be a patients 
   waiting  for treatment area specified, for every patients waiting for 
   treatment  area a casualties clearing station and for every ampulance 
   parking point a casualties clearing station.
\end{itemize}


\section{Converting scenarios to other formats}

The native format in which BonnMotion saves the movement traces is
node-by-line waypoint based. This means that there is one line for
each node. This line contains all the waypoints. A waypoint is a
position at which the movement of a node (e.g. direction, velocity)
changes. A waypoint consists of:
o  the simulation time in seconds at which the waypoint is reached 
   by the node
o  the x and y coordinates of the position of the waypoint


\subsection{ns-2}


The  "NSFile"  application  is  used  to  generate two files that can be
integrated into a TCL  script  to  start  an  ns-2  simulation  via  the
"source" command.

The  file with the suffix ``.ns\_params'' sets some variables needed to set
up the simulation in an array named ``val'': the keys ``x'' and ``y'' refer to
width  and  height  of the simulation area, "nn" refers to the number of
nodes and "duration" to the duration of the simulation.

The file with the suffix ".ns\_movements" schedules the movements of  the
node objects that are expected to be in an array named "node\_", numbered
starting at 0. The simulator object is expected to be  in  the  variable
"ns\_".

As  a  side  note,  the "NSFile" application places an additional margin
around the simulation area, because ns-2 versions up to 2.1b9a regularly
crash  when  nodes move at the border of the simulation area. (Actually,
this has only been observed with the Manhattan Grid model up to now, but
this procedure is generally carried out just to play it safe.)
\newline \newline
Usage example:\newline

   bm NSFile -f scenario1
\newline \newline
creates      the      two      files      "scenario1.ns\_params"      and
"scenario1.ns\_movements".


\subsection{Glomosim / Qualnet}


The  "GlomoFile"   application   creates   files   with   the   suffixes
".glomo\_nodes"  and  ".glomo\_mobility",  which can be used with Glomosim
(2.0.3) and Qualnet (3.5.1). Use  the  "-q"  switch  for  Qualnet:  This
causes nodes to be numbered starting at 1, not at 0.


\subsection{XML}


The "SPPXml" application is used to generate mobility files in XML
format according to the XML schema proposed by Horst Hellbrück as a
standardised mobility file format for the research program
``Schwerpunktprogramm 1140''
(\url{http://www.tm.uka.de/forschung/SPP1140/}) of the DFG (Deutsche
Forschungsgemeinschaft).

SPPXml has an additional parameter "-r" that allows to specify a uniform
radio  range  for all nodes. If this parameter is omitted, a radio range
of 250m is used. BonnMotion does not use this radio range in any way. It
is however required by the XML schema.

The  XML  schema corresponding to the XML files generated by "SPPXml" is
defined in:

\url{http://www.i-u.de/schools/hellbrueck/ansim/xml/sppmobtrace.xsd}

The  contrib   directory   furthermore   includes   a   python   program
"sppmob2bonnmotion.py"   that   convertw  XML  mobility  traces  to  the
BonnMotion format. It has been tested with Python 2.3.


\subsection{IntervalFormat}

The native format implies that during the simulations for each event
the current node positions have to be calculated based on the
waypoints.  If there are many events, this may have a negative impact
on the runtime of a simulation. An alternative is to use an interval
based approach. The nodes are regarded as stationary for an
interval. The positions of the nodes are updated periodically after
each interval by a specific position update event. By doing so, the
current node positions do not have to be calculated for each
event. However, the number events is increased, which may also
influence the runtime of a simulation negatively. A factor that has a
major impact in this context is the interval length. Smaller intervals
yield higher accuracy but also more events. Overall, it is a trade-off
between the number of events and the runtime per event.

Trace files in the BonnMotion's native trace format can be transformed
to an interval-based format using the IntervalFormat application. The
interval length can be specified using the -l option. The default
value is one second. The interval trace format is an interval-by-line
based. This means that there is one line for each interval of each
node. A line consists of: 
\begin{itemize}
\item the node number
\item the simulation time in seconds (in intervals) 
\item the x and y coordinates of the position of the node for the interval 
\end{itemize}

The IntervalFormat application prints the waypoints (ordered by node
and time) for every interval step.
\begin{itemize}
\item The used interval can be specified using the -l switch.
\item Using the -s switch the header can be skipped
\end{itemize}


\section{Scenario analysis}

\subsection{The Statistics application}


The "Statistics" application is used to analyse a given scenario.  There
are  two  modes  of  operation:  The  default  is to calculate "overall"
statistics (averaged over the simulation time) and the other mode is  to
calculate "progressive" statistics (values of metrics for certain points
in time).

In its default mode, the application creates  a  file  with  the  suffix
".stats" containing the following information:

\begin{itemize}
\item The  \emph{relative mobility} is independent of the transmission range and
   is calculated according to \cite{johansson_scenario_based_1999}.

\item The  other  figures are dependent on the transmission range, which is
   given in the first column of every line.

\item The second column gives the average node degree: To how many  of  the
   other nodes is one node connected?

\item The  third column gives the average number of partitions: 1 means the
   network is connected at all times, values larger than 1 indicate that
   this is not the case.

\item The  fourth  column  gives the degree of separation: How likely is it
   that two randomly chosen nodes are within a connected component at  a
   randomly chosen point in time?

\item The  fifth column gives the average link duration; only links that go
   up after the simulation start and go down before the  simulation  end
   are taken into account.

\item The  sixth  column  gives  the  standard  deviation  of  the previous
   measure.

\item The seventh column gives  the  number  of  links  considered  in  the
   calculation of the two previous values.

\item The  eighth  column is like the fifth, but this time, \_all\_ links are
   counted.

\item The ninth column gives the total number of links.
\end{itemize}

Alternatively to the  statistics  which  are  averaged  over  the  whole
simulation  time,  the  devolution  of  certain  characteristics  can be
calculated. See the command line help and the following examples.
\newline \newline
Usage examples:
\newline

   bm Statistics -f scenario1 -r 50,75,100
\newline \newline
This  writes  the   averaged   statistics   to   "scenario1.stats"   for
transmission ranges of 50, 75, and 100 meters.
\newline

   bm Statistics -f scenario1 -r 75 -P
\newline \newline
This  creates  a file scenario1.stats\_75.part, which gives the number of
partitions each time it changes.
\newline

   bm Statistics -f scenario1 -r 50,100 -N -M 10
\newline \newline
This    creates    the    files     "scenario1.stats\_50.nodedeg"     and
"scenario1.stats\_100.nodedeg"  which  show  the  devolution  of the node
degrees.  Furthermore,   the   files   "scenario1.stats\_50.mincut"   and
"scenario1.stats\_100.mincut"  show the minimal cut of the topology every
10  seconds.  It  is  reasonable  to  specify  such  a  time  span   for
computations that cost much CPU time.


\subsection{The LinkDump application}


The  LinkDump  application  prints  information  about every single link
within a certain time span of a scenario. This information can  e.g.  be
used for a simulator that does not directly model mobility at all.

Using  the  -d  switch, only the link durations are printed. This offers
more insight into the distribution of link durations than  the  averaged
value  calculated  by  the  Statistics application. In this case, the -w
switch prevents that wrong durations are printed out due to  links  that
go up before simulation begin or go down after simulation end.


\subsection{The Dwelltime application}

The Dwelltime application creates statistics how long nodes stay in which 
area of the simulated region.


\section{Scenario visualisation}

"Visplot"  is a very simple application that writes those positions to a
file where a mobile changes its speed or direction. This file can simply
be visualised using e.g. gnuplot.


\section{Acknowledgments}
The first versions of this software were designed and implemented by
Michael Gerharz and Christian de Waal. The development of these
versions were supported in part by the German Federal Ministry of
Education and Research (BMBF) as part of the IPonAir project.

Since 2008 BonnMotion is partially supported by CONET, the Cooperating
Objects Network of Excellence, funded by the European Commission under
FP7 with contract number FP7-2007-2-224053.

Further extension have been designed and implemented by Nils
Aschenbruck, Raphael Ernst, Elmar Gerhards-Padilla, Tim Heinrich,
Patrick Peschlow, and Matthias Schwamborn.

\clearpage

\addcontentsline{toc}{section}{References}
\bibliographystyle{natdin}
\bibliography{references}

\end{document}
